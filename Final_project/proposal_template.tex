%
% File proposal_template.tex
%
%% Based on the style files for ACL 2018, NAACL 2018/19, which were
%% Based on the style files for ACL-2015, with some improvements
%%  taken from the NAACL-2016 style
%% Based on the style files for ACL-2014, which were, in turn,
%% based on ACL-2013, ACL-2012, ACL-2011, ACL-2010, ACL-IJCNLP-2009,
%% EACL-2009, IJCNLP-2008...
%% Based on the style files for EACL 2006 by 
%%e.agirre@ehu.es or Sergi.Balari@uab.es
%% and that of ACL 08 by Joakim Nivre and Noah Smith

\documentclass[11pt,a4paper]{article}
\usepackage[hyperref]{acl2019}
\usepackage{times}
\usepackage{latexsym}

\usepackage{url}

\aclfinalcopy % Uncomment this line for the final submission
%\def\aclpaperid{***} %  Enter the acl Paper ID here

%\setlength\titlebox{5cm}
% You can expand the titlebox if you need extra space
% to show all the authors. Please do not make the titlebox
% smaller than 5cm (the original size); we will check this
% in the camera-ready version and ask you to change it back.

\newcommand\BibTeX{B\textsc{ib}\TeX}

\title{Reproducibility check Study for [PAPER]}

\author{First Author \\
  Affiliation / Address line 1 \\
  Affiliation / Address line 2 \\
  Affiliation / Address line 3 \\
  \texttt{email@domain} \\\And
  Second Author \\
  Affiliation / Address line 1 \\
  Affiliation / Address line 2 \\
  Affiliation / Address line 3 \\
  \texttt{email@domain} \\}

\date{}

\begin{document}
\maketitle

\section{Introduction}

Should explain the context of the paper. It should contain the following subsections:

            \subsection{Task / Research Question Description} The task was to develop a tool that can securely store and verify user passwords using unique salts and slow hashing techniques without using the Bcrypt library. The implementation needed to focus on security best practices for password storage.

            \subsection{Motivation \& Limitations of existing work} 
            Secure password storage remains a critical challenge in application security. Several approaches to password storage have been developed over time, each with significant limitations:
            \begin{itemize}
              \item \textbf{Plain text storage:} Historically, many applications stored passwords in plain text. This approach offers no security if the database is compromised, leading to immediate exposure of all user credentials.
              
              \item \textbf{Simple hash functions (MD5, SHA-1, SHA-256):} These general-purpose cryptographic hash functions were not designed specifically for password storage. They are computationally inexpensive, making brute-force attacks feasible with modern hardware. Additionally, they lack built-in salt management, making them vulnerable to rainbow table attacks.
              
              \item \textbf{Salted hashes:} Adding salt to passwords before hashing improves security, but many implementations use fixed or predictable salts, reducing their effectiveness against precomputed attacks.
              
              \item \textbf{Fast hash iterations:} Some implementations attempt to increase security by repeatedly hashing the password, but without a properly designed algorithm, this approach can be inefficient and still vulnerable to hardware-accelerated attacks.
              
              \item \textbf{Bcrypt:} While Bcrypt is widely recommended for password hashing, it has several limitations including a 72-byte password length cap and inconsistent implementations across platforms. Additionally, the task specifications prohibit using Bcrypt for this implementation.
            \end{itemize}
            
            Modern password storage requires features that address these limitations, including unique cryptographically strong salts, computationally expensive hashing algorithms specifically designed for password storage, and configurable parameters to adapt to increasing computational power over time.            

            \subsection{Implementation Approach} 
            The implementation uses the Web Cryptography API with PBKDF2 (Password-Based Key Derivation Function 2) and SHA-256 hashing. The approach includes:
            \begin{itemize}
              \item Generating unique random salts for each password
              \item Applying PBKDF2 with configurable iteration counts
              \item Storing password hashes, salts, and metadata in localStorage
              \item Measuring and displaying performance statistics for different security parameters
            \end{itemize}
            \subsection{Likely challenges and mitigations} 
            INCLUDE AT THE END 


\section{Related Work}
Password-based authentication remains the predominant method for securing user accounts despite its known limitations. \cite{bonneau2012quest} conducted a comprehensive analysis of authentication schemes, concluding that password-based schemes, while problematic, continue to dominate due to their deployability advantages. For password storage specifically, \cite{turan2018recommendation} proposed standardized methods for secure password hashing, emphasizing the importance of key derivation functions with tunable work factors like PBKDF2, which is implemented in our study application.
The performance-security tradeoff in password hashing has been examined by \cite{visconti2020evaluate}, who evaluated various password hashing schemes across different platforms, demonstrating how computational costs vary significantly based on implementation choices. Their work provides valuable benchmarks for assessing the efficiency claims in our target paper. Similarly, \cite{pesante2021empirical} conducted an empirical study of client-side password hashing performance, particularly relevant to our web-based implementation that performs cryptographic operations in the browser. Our work differs from these studies by specifically examining the reproducibility of performance claims made in the original paper about PBKDF2 implementation in browser environments. Additionally, we extend previous work by analyzing the robustness of the password security implementation against varying client hardware capabilities, an aspect often overlooked in theoretical security analyses but critical for real-world deployments.

\section{Experiments}

\subsection{Implementation} 
Please provide a link to a repo of your reimplementation (if applicable) and appropriately cite any resources you have used.

\subsection{Results}
Provide a table comparing your results to the published results.

\subsection{Discussion}
Discuss any issues you faced. Do your results differ from the published ones? If yes, why do you think that is? Did you do a sensitivity analysis (e.g. multiple runs with different random seeds)?

\section{Robustness Study}

Robustness in our password management system refers to its ability to maintain security and functionality under adverse conditions or attacks. The system implements cryptographic best practices including PBKDF2 with SHA-256, unique salts, and configurable iterations to protect against common attack vectors such as brute force and rainbow table attacks. Robustness is critical for authentication systems as they are primary targets for attackers - a compromised system can lead to unauthorized access to sensitive data. Our approach balances strong security with acceptable performance to create a resilient system that protects user credentials while remaining usable.

\subsection{Password Security Implementation}
Our password management system implements several cryptographic mechanisms to ensure robust credential protection:
\begin{itemize}
  \item \textbf{PBKDF2 with SHA-256:} The system leverages PBKDF2 with SHA-256 to derive cryptographic keys from passwords, providing resistance against hardware-accelerated attacks.
  \item \textbf{Unique Salt Generation:} Each password receives a 16-byte random salt generated via the Web Crypto API, effectively preventing precomputed attacks and ensuring identical passwords produce different hashes.
  \item \textbf{Configurable Iteration Counts:} Users can adjust PBKDF2 iterations (10,000-500,000) through a slider interface, allowing customization of the security-performance tradeoff. Higher iterations increase computational complexity for attackers.
  \item \textbf{256-bit Key Length:} The derived 256-bit keys provide substantial security margin against future advances in computational capabilities.
\end{itemize}

This implementation follows current cryptographic best practices while offering flexibility to adapt to different security requirements and computing environments.

\subsection{Error Handling and Input Validation}

The password management system implements comprehensive error handling and input validation to maintain robustness against both accidental misuse and deliberate attacks:

\begin{itemize}
   \item \textbf{Form Validation:} All input forms require essential fields to be completed before submission through HTML5's ``required'' attribute, preventing incomplete data from being processed.
   
   \item \textbf{Duplicate Username Detection:} The system proactively checks for existing usernames during registration, providing clear error messages and preventing accidental or malicious overwrites of existing accounts.
   
   \item \textbf{Informative Error Messages:} The system delivers contextual error messages that are informative enough to guide legitimate users while avoiding excessive information disclosure that could aid attackers.
   
   \item \textbf{Failed Authentication Handling:} Login and password change operations verify credentials before performing sensitive actions, with standardized ``Invalid username or password'' messages that don't reveal which specific credential was incorrect.
   
   \item \textbf{Visual Feedback:} Color-coded status messages (green for success, red for errors) provide immediate visual feedback on operation outcomes, enhancing user awareness of system state.
   
   \item \textbf{Timeout-based Message Clearing:} Status messages automatically clear after a short period, reducing the window for shoulder-surfing attacks and maintaining a clean interface.
\end{itemize}

These measures ensure the system remains stable and secure when handling both valid and invalid inputs, contributing significantly to overall robustness.

\subsection{Performance Considerations}

The password management system incorporates performance measurement and configuration options that allow for explicit security-performance tradeoffs:

\begin{itemize}
    \item \textbf{Configurable PBKDF2 Iterations:} The system implements user-configurable iteration counts ranging from 10,000 to 500,000 (with increments of 10,000), allowing users to explicitly balance security strength against computational cost. The default setting is 100,000 iterations.
    
    \item \textbf{Real-time Performance Metrics:} The system measures and displays actual performance metrics using the browser's high-resolution \texttt{performance.now()} API, capturing millisecond-precision timing for all cryptographic operations.
    
    \item \textbf{Operation-specific Measurements:} Distinct performance measurements are tracked for:
    \begin{itemize}
        \item Password registration (key derivation time)
        \item Login verification (key derivation and comparison time)
        \item Password changes (comparing performance between old and new password configurations)
    \end{itemize}
    
    \item \textbf{Performance Feedback:} Users receive immediate performance feedback after each operation, displaying the operation duration in milliseconds alongside the iteration count used, creating awareness of the security-performance relationship.
    
    \item \textbf{Comparative Analysis:} When changing passwords, the system provides side-by-side comparison between previous and new password performance metrics, enabling users to make informed decisions about security configurations.
\end{itemize}

These performance features enable a transparent, user-controlled approach to the fundamental security-performance tradeoff inherent in password hashing, allowing appropriate configurations based on specific security requirements and available computational resources.

\subsection{Potential Vulnerabilities and Mitigations}

Despite the implementation of modern cryptographic techniques, several potential vulnerabilities exist in both the system architecture and the cryptographic methods employed:

\begin{itemize}
   \item \textbf{PBKDF2 Limitations:} While PBKDF2 with SHA-256 is a standard approach, it remains vulnerable to hardware acceleration attacks using GPUs or ASICs. More memory-intensive key derivation functions like Argon2 or scrypt would provide stronger resistance against specialized hardware attacks by requiring significant memory alongside computational resources.
   
   \item \textbf{Fixed Salt Length:} The implementation uses a 16-byte salt, which meets minimum standards but could be strengthened. Increasing the salt length to 32 bytes would provide additional protection against precomputation attacks without significant performance impact.
   
   \item \textbf{Insufficient Minimum Iterations:} The system allows iteration counts as low as 10,000, which may be insufficient against modern hardware capabilities. OWASP recommends a minimum of 310,000 iterations for PBKDF2-HMAC-SHA256. The slider should enforce higher minimum values to prevent users from selecting dangerously low settings.
   
   \item \textbf{Browser Crypto API Limitations:} The Web Crypto API implementation may vary across browsers and could potentially use different PBKDF2 parameters or pseudorandom number generators with varying security properties. Validation testing across multiple browsers would help identify inconsistencies.
   
   \item \textbf{Side-Channel Vulnerability:} The precise performance timing displayed after operations could potentially leak information about password characteristics through timing side channels. Implementing constant-time comparison operations and adding random delays would help mitigate this class of attacks.
\end{itemize}

These cryptographic vulnerabilities complement the architectural concerns about client-side security and localStorage usage, presenting a more comprehensive view of the system's security posture and areas for improvement.

\subsection{Conclusion and Future Improvements}

The password management system demonstrates several robust security features including PBKDF2 with SHA-256 hashing, unique cryptographic salts, configurable iteration counts, comprehensive error handling, and transparent performance metrics. These elements collectively create a system that balances security requirements with performance considerations while providing clear user feedback.

\subsection{Suggested Enhancements}
Future improvements to increase robustness should focus on:
\begin{itemize}
   \item Moving critical security operations to a server-side implementation with secure transport
   \item Adding brute force protection through rate limiting and account lockout mechanisms
   \item Increasing minimum iteration counts to align with current OWASP recommendations
   \item Implementing secure credential storage using HTTP-only cookies or server-side databases
\end{itemize}

These enhancements would address the identified vulnerabilities while maintaining the system's current strengths in usability and security transparency.

\subsection{Results of Robustness Evaluation}

% This section would contain your actual findings from running the robustness tests

\subsubsection{Success Cases}

Here we present two examples where the model demonstrated impressive robustness:

\paragraph{Example 1:} [Description of input transformation and why model performance remained stable]

\paragraph{Example 2:} [Another example showing resilience to a different type of perturbation]

\subsubsection{Failure Cases}

We also identified two significant failure modes that indicate areas for improvement:

\paragraph{Example 1:} [Description of perturbation that caused dramatic performance degradation]

\paragraph{Example 2:} [Another example highlighting a different vulnerability]

\subsection{Discussion}

Our robustness evaluation revealed several interesting patterns that extend beyond the findings reported in the original paper. First, the model demonstrates significant sensitivity to [specific type of perturbation], suggesting that [specific component] may be a weakness in the architecture. Second, performance degrades non-linearly with increasing resource constraints, with a sharp decline occurring at [specific threshold].

These findings highlight the importance of comprehensive robustness testing beyond standard benchmarks. If we had more time, we would explore [additional robustness dimensions] and investigate potential mitigation strategies for the identified vulnerabilities.

For researchers conducting similar robustness analyses, we recommend:
\begin{enumerate}
    \item Creating systematic perturbation hierarchies with controlled difficulty levels
    \item Testing incrementally to isolate specific failure points
    \item Maintaining a diverse set of success and failure examples to guide improvement
    \item Considering deployment constraints early in the evaluation process
    \item Developing standardized robustness metrics that complement traditional accuracy measures
\end{enumerate}

Such comprehensive evaluation is essential for moving beyond headline performance numbers toward models that function reliably in diverse, unpredictable real-world environments.

\subsection{Results of Robustness Evaluation}
Describe the evaluation results of your reproduced model on the robustness benchmark that you created. Include at least 2 examples where the model performs well and 2 examples where it fails (i.e., being not robust). Provide sufficient analysis and your thoughts on the observations.

\subsection{Discussion} 
Provide any further discussion here, e.g., what challenges did you face when performing the analysis, and what could have been done if you will have more time on this project? Imagine you are writing this report to future researchers; be sure to include "generalizable insights" (e.g., broadly speaking, any tips or advice you'd like to share for researchers trying to analyze the robustness of an NLP model).

\section{Workload Clarification}

Our team approached this reproducibility study through a balanced division of responsibilities, ensuring equal contribution from all members. The workload distribution was organized as follows:

\begin{itemize}
    \item \textbf{Implementation and Code Development:} Team member 1 took primary responsibility for setting up the code environment, implementing the core algorithms described in the paper, and ensuring that our implementation matched the original authors' description. This included debugging runtime errors and optimizing performance.
    
    \item \textbf{Experimental Design and Data Collection:} Team member 2 designed the experimental protocol, created the evaluation datasets, and executed the main experiments. This involved configuring the test environments, collecting performance metrics, and organizing the results for analysis.
    
    \item \textbf{Robustness Testing:} Team member 3 developed the robustness evaluation framework, designed the perturbation types, and conducted all robustness experiments. This included creating specialized test cases, implementing adversarial scenarios, and analyzing model behavior under various constraints.
    
    \item \textbf{Analysis and Documentation:} All team members collaborated on analyzing experimental results, with each member focusing on their area of responsibility. Team member 1 analyzed implementation challenges, team member 2 assessed performance results against published claims, and team member 3 evaluated robustness findings.
\end{itemize}

Throughout the project, we maintained regular team meetings to discuss progress, address challenges, and align our understanding of the paper. While we maintained these primary responsibilities, we frequently assisted each other across areas to ensure comprehensive coverage and shared understanding of all aspects of the reproducibility study.

All team members contributed equally to the final writing of this report, with each member drafting sections related to their primary responsibilities and collectively reviewing and refining the complete document.

\section{Conclusion}
INSERT

% \section{Credits}

% This document has been adapted from the instructions
% for earlier ACL and NAACL proceedings,
% including 
% those for 
% NAACL 2019 by Stephanie Lukin and Alla Roskovskaya, 
% ACL 2018 by Shay Cohen, Kevin Gimpel, and Wei Lu, 
% NAACL 2018 by Margaret Michell and Stephanie Lukin,
% 2017/2018 (NA)ACL bibtex suggestions from Jason Eisner,
% ACL 2017 by Dan Gildea and Min-Yen Kan, 
% NAACL 2017 by Margaret Mitchell, 
% ACL 2012 by Maggie Li and Michael White, 
% those from ACL 2010 by Jing-Shing Chang and Philipp Koehn, 
% those for ACL 2008 by JohannaD. Moore, Simone Teufel, James Allan, and Sadaoki Furui, 
% those for ACL 2005 by Hwee Tou Ng and Kemal Oflazer, 
% those for ACL 2002 by Eugene Charniak and Dekang Lin, 
% and earlier ACL and EACL formats.
% Those versions were written by several
% people, including John Chen, Henry S. Thompson and Donald
% Walker. Additional elements were taken from the formatting
% instructions of the \emph{International Joint Conference on Artificial
%   Intelligence} and the \emph{Conference on Computer Vision and
%   Pattern Recognition}.

\bibliographystyle{acl_natbib} % We choose the "plain" reference style
\bibliography{refs} % Entries are in the refs.bib file

@article{bonneau2012quest,
  title={The quest to replace passwords: A framework for comparative evaluation of web authentication schemes},
  author={Bonneau, Joseph and Herley, Cormac and Van Oorschot, Paul C and Stajano, Frank},
  journal={2012 IEEE Symposium on Security and Privacy},
  pages={553--567},
  year={2012},
  publisher={IEEE}
}

@article{turan2018recommendation,
  title={Recommendation for password-based key derivation: Part 1: Storage applications},
  author={Turan, Meltem S and Barker, Elaine and Burr, William and Polk, Tim and Smid, Miles},
  journal={NIST Special Publication},
  volume={800},
  number={132},
  year={2018}
}

\end{document}
